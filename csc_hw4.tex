\documentclass[10pt,a4paper]{article}
\usepackage[utf8]{inputenc}

\title{%
  COMSOC: Homework 4 \\
  \large Silvan Hungerbuehler, 11394013}

\usepackage{mathptmx} % "times new roman"
\usepackage{amssymb}
\usepackage[fleqn]{amsmath}
\usepackage{amsthm}
\usepackage{amsfonts}
\usepackage{enumitem}
\usepackage{verbatim}
\usepackage{hyperref}
\usepackage{comment}
\usepackage[margin=1in]{geometry}
\usepackage{float}
\usepackage{bm}
\usepackage{graphicx}
\graphicspath{ {/home/sh/Desktop/} }

\usepackage[normalem]{ulem}
\date{}
\begin{document}
\maketitle
\section*{Question 1}
Basic idea: Given a special candidate $x^*\in X$ and a partial profile $\mathbf{R}\in \mathcal{P}(X)^n$, we compute how many first positions $x^*$ minimally obtains and how many first positions any of the other candidates can maximally obtain for all the ways $\mathbf{R}$ can be completed. If the minimal number of first positions of $x^*$ is still higher than the maximal number of any other candidate, then $x^*$ is a \textit{necessary winner}, otherwise not.\\
\underline{Algorithm}: \\
Set up a score counter for all candidates. All candidates have $0$ points initially. \\
For all individual ballots $r\in\mathbf{R}$, if $x^*$ is a single winner, award her $1$ point. A candidate is a single winner in a profile, if she is in the first position and strictly preferred to the second candidate in the ballot. Note that this requires checking at most two things in the ballot: who is in the first position and the relational symbol ($\succ$ or $\sim$) with respect to the second.\\
If $x$ is no single winner, award all other candidates in the top positions $1$ point. A candidate is in the top position in a given ballot, if there no alternative is preferred over her. Note that this requires checking the positions of the ballot until encountering the first clear preference of one over the other. All the candidates before the first $\succ$ symbol obtain $1$ point, except for $x^*$.\\
After having run through all the $r\in \mathbf{R}$, if $x^*$ has the sole maximal score, it is the \textit{necessary winner}, otherwise not \footnote{Let's assume, for simplicity, that ties are broken randomly. Thus, being tied for most points cannot garantuee victory.}.\\

\noindent The algorithm runs in polynomial time becaus it needs to go through each individual ballot at most once to award points. That is, it need not take more steps than there are candidates for each ballot. Also, it needs to do this for all voters only once. Computing the maximum amongst the scores also takes no more steps than the number of candidates (compare any two, discard the lower and keep the higher (if tied keep both), compare it against the next etc., the last remaining one is the maximum).

\section*{Question 2}
Consider the following profile consisting of two voters' preferences. I show below that it creates a circle for iterative voting under the Borda rule and best responses. Iterative voting thus need not converge for the Borda rule and best responses:\\
$\succ_1=(A\succ B \succ C)$\\
$\succ_2=(B\succ C \succ A)$\\
Now consider the following sequence of best responses that the two voters make. $t_i$ denotes the voting round ($t_0$ being the initial round with true preferences), $BS(X)$ the Borda score of candidate $X$. Stars above the reported preferences indicate a change.
\begin{align*}
&t_0 & &t_1\\
& \succ_1=(A\succ B \succ C) &  &\succ_1^*=(A\succ C \succ B) \\
& \succ_2=(B\succ C \succ A) && \succ_2=(B\succ C \succ A)\\
&BS(A)=2,BS(B)=3,BS(C)=1, Winner: B &&BS(A)=2,BS(B)=2,BS(C)=2, Winner: A\\
\end{align*}
Voter $1$ changed her true ballot $\succ_1$ to $\succ_1^*$ in order to make her preferred option $A$ win (by lexicographical ordering) instead of $B$.
\begin{align*}
&t_2 &&t_3\\
& \succ_1^*=(A\succ C \succ B) & &\succ_1^{**}=(A\succ B \succ C) \\
& \succ_2^*=(C\succ B \succ A) && \succ_2^*=(C\succ B \succ A)\\
&BS(A)=2,BS(B)=1,BS(C)=3, Winner: C &&BS(A)=2,BS(B)=2,BS(C)=2, Winner: A\\
\end{align*}
In $t_2$ the voter $2$ changed her ballot, so $C$, which she prefers to $A$, would win. In $t_3$ voter $1$ changed her ballot again, so $A$ would win again.
\begin{align*}
&t_4\\
& \succ_1^{**}=(A\succ B \succ C)  \\
& \succ_2^{**}=(B\succ C \succ A) \\
&BS(A)=2,BS(B)=3,BS(C)=1, Winner: B\\
\end{align*}
Finally, voter $2$ changes her ballot again in $t_4$ and $B$ wins. The profile has come full circle, as $t_0=t_4$.

\section*{Question 3}
First, observe that if each voter only specifies one goal $\phi$, then this induces a preference order for this voter that neatly divides the candidates in two groups: One group in which all candidates satisfy $\phi$ and are related by $\sim$ with each other, call this the group for voter $i$ $top_i$; and another group of candidates that all fail to satisfy $\phi$, call them $bottom_i$. The candidates in $bottom_i$ are also related by $\sim$ amongst each other but they are dispreferred to all candidates in $top_i$. Thus, voter $i$'s preference order induced by her sole goal $\phi$ looks something like this:\\
$x_1\sim ... \sim x_l \succ x_{l+1} \sim ... \sim x_m$, where $top_i=\{x_1,...,x_{l}\}$ and $bottom_i=\{x_{l+1},...,x_m\}$ for $X=\{x_1,...,x_{l+1},...,x_m\}$\\
For $n$ many voters with one goal each we thus have $n$ many preference orders, and each preference order contains one $top$ group of candidates and one $bottom$ group of candidates. Two cases remain to be dealt with: If none or all candidates satisfy the goal, then we put all candidates in the $top$ group. We can exploit this fact to always find at least one \textit{weak Condorcet winner}. This works as follows:\\
For each voter $i\in N$ assign each candidate in $top_i$ one point and each candidate in $bottom_i$ zero points. This essentially counts how many times a candidate is undominated over the whole profile. That is, how many times the candidate is dispreferred against no other candidate in the preference orders induced by the goals. Now, notice that, by definition, a candidate $x$ is a \textit{weak Condorcet winner} exactly if no other candidate dominates - is preferred to her - more times over the profile. To see this, assume that $x^*$ is a \textit{weak Condorcet winner} and assume towards a contradiction that she does not have the maximal count. That means that there must be another candidate who was in the $top$ group in more preference orders than $x^*$. But since both the other candidate and $x^*$ are present in each preference order and either in $top$ or $bottom$ that means that the other candidate won the pairwise majority contest against $x^*$. Hence, $x^*$ could not have been a \textit{weak Condorcet winner}. We can, therefore, simply look at each candidate's count and elect the candidate/s with the maximal count/s.
\end{document}