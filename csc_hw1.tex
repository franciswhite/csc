\documentclass[10pt,a4paper]{article}
\usepackage[utf8]{inputenc}

\title{%
  COMSOC: Homework 1 \\
  \large Silvan Hungerbuehler, 11394013}

\usepackage{mathptmx} % "times new roman"
\usepackage{amssymb}
\usepackage{amsmath, amsthm}
\usepackage{amsfonts}
\usepackage{enumitem}
\usepackage{verbatim}
\usepackage{hyperref}
\usepackage{comment}
%\usepackage[margin=1in]{geometry}
\usepackage{float}
\usepackage{bm}

\usepackage[normalem]{ulem}
\date{}
\begin{document}
\maketitle

\section*{Question 1}
\subsection*{a}
$N=\{1,2,3,4,5\}\\
X=\{A,B,C,D\}\\
\boldsymbol{\succ}=(\succ_1,...,\succ_5)\\
\succ_1=\succ_2=(A \succ B \succ C \succ D)\\
\succ_3=(B \succ C \succ D \succ A)\\
\succ_4=(C \succ D \succ B \succ A)\\
\succ_5=(D \succ B \succ C \succ A)$
$A$ wins the plurality vote because it is most preferred by two agents. Since it is the last choice for three other agents, $A$ will lose all pairwise contests.
\subsection*{b}

\section*{Question 2}
\subsection*{a}
Moulin, H. 1988b. Condorcet’s Principle Implies the No-Show Paradox. Journal of Economic Theory,
45, 53–64.
\subsection*{b}
I opened the Handbook, control-F'd \textit{'Moulin'}, went to the References Section where three papers fit the time frame: one of them bore the right title. I then google'd the title to check the abstract and be certain.
\subsection*{c}
Precise statement of Statement using notation from class:\\
voting rule: f
\subsection*{d}
Political Scientist: \href{http://link.springer.com/chapter/10.1007/978-3-642-20441-8_3}{click}\\
CS: \href{http://link.springer.com/chapter/10.1007%2F978-3-642-21581-0_25}{click}\\
Philosophy: \href{http://citeseerx.ist.psu.edu/viewdoc/download?doi=10.1.1.677.4121&rep=rep1&type=pdf}{click}


\end{document}