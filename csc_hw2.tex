\documentclass[10pt,a4paper]{article}
\usepackage[utf8]{inputenc}

\title{%
  COMSOC: Homework 2 \\
  \large Silvan Hungerbuehler, 11394013}

\usepackage{mathptmx} % "times new roman"
\usepackage{amssymb}
\usepackage{amsmath, amsthm}
\usepackage{amsfonts}
\usepackage{enumitem}
\usepackage{verbatim}
\usepackage{hyperref}
\usepackage{comment}
\usepackage[margin=1in]{geometry}
\usepackage{float}
\usepackage{bm}

\usepackage[normalem]{ulem}
\date{}
\begin{document}
\maketitle

\section*{Question 1}
\subsection*{a}
The antiplurality rule is characterized by the \textit{discrete distance} and the \textit{unanimous loser} consensus criterion.\\
A candidate $x$ is a \textit{unanimous loser} with respect to some profile iff all voters rank $x$ last in their individual ballot.
\subsection*{b}
\textit{Cindarella-Swap Distance}:\\\
This measure of distance between two profiles - $\boldsymbol{R}$ and $\boldsymbol{R'}$ -  only takes into account swaps from one end of an individual ranking to another. The distance is the minimal number of such extremely-opposed swaps needed to turn $\boldsymbol{R}$ into $\boldsymbol{R'}$ or vice versa.
The antiplurality rule is characterized by the \textit{unanimous winner} consensus criterion and the maximal \textit{Cindarella-Swap Distance}.
\subsection*{c}
Noise Model: Each voter independently ranks the true winner of the antiplurality rule at the $m^{th}$ position with probability $1-\epsilon$. All other candidates occupy other positions with probability $\tfrac{\epsilon}{m-1}$ each.\\
Then a candidate $x$ has maximum likelihood of being the true winner iff $x$ is the antiplurality winner.
\section*{Question 2}
\subsection*{a}
How many voting rules?\\
For $x$ candidates and $m$ voters: There are $x!$ many ways to order the candidates in an individual ballot, so there are $x!^m$ possible profiles ($m$ 'spots' and $x!$ options in each 'spot'). This is the cardinality of the domain of our social choice function (SCF). The cardinality of the codomain is simply $2^x-1$ - the cardinality of the power set minus the empty set. Thus there are $[2^x-1]^{x!^m}$ SCF in general. For $m=n=3$ this makes $[2^3-1]^{3!^3}=7^{216}$ many SCF. \\
How many of them anonymous?\\
There are now a mere $x!*m$ elements in the domain to be considered (because a different ordering of individual rankings does not imply a different profile anymore). Therefore, there are $[2^x-1]^{x!*m}$ SCF in general; particularly, $[2^3-1]^{3!*3}=7^{18}$ in our case.\\
How many neutral?\\ 
How many anonymous and neutral?\\
How many are resolute?\\
The codomain's cardinality would be equal to the number of candidates: $x$. So there are $x^{(x!^m)}$ possible SCF in general, or $3^{(3!^3)}=3^{216}$ in our case.
How many are resolute and anonymous?\\
Combining the domain's cardinality under anonymity with the resolute codomain already determined yields generally $x^{(x!*m)}$ possible SCF - this makes $3^{(3!*3)}=3^{18}$ for $m=n=3$.\\
How many are resolute and neutral?\\
How many are resolute, neutral and anonymous?\\



\section*{Question 3}
\subsection*{a}
It is impossible to prove the claim that some rule is always strictly the best one to minimize the \textit{average regret}. The \textit{average regret} depends on the specific preference profile one is looking at and there are such profiles where all rules perform identically.\\

\noindent Consider the following situation where $N=\{1,2\}$ and $X=\{A,B\}$ and the preference profile $\boldsymbol{\succ}=(\succ_1,\succ_2)$ is as follows:\\
$\succ_1=(A\succ B)$\\
$\succ_2=(B\succ A)$\\
Any resolute voting rule will either declare $A$ or $B$ the winner. In either case the average regret is equal to $0.5$, form one player will have an individual regret of $0$ while the other has an individual regret of $1$.\\
Thus there is no meaningful sense in which any voting rule generally is the 'best choice' in terms of average regret minimisation; given this specific preference profile they all perform equally well.
\subsection*{b}
No matter which $x\in X$ wins, agent $i$'s preference order is uniformly (and independently from all other agents, I assume) drawn from $m!$ possible such orderings. There are therefore equally many orderings where $x$ occupies the first place, the second place, the third etc. in $\succ_i$. Each place of $x$ in $i$'s order is thus equally probable. Hence, each regret-value - from $m-1$ to $0$ is equally probable; that is, has probabality $\tfrac{1}{m}$. \\
Under this assumption of uniform distribution of individual regret, and as $n$ grows very large, the mean or average of the $n$ individual regret-values will converge to the \textit{average expected regret} by the weak law of large numbers. It will simply be $\tfrac{m-1}{2}$.
\end{document}