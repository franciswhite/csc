\documentclass[10pt,a4paper]{article}
\usepackage[utf8]{inputenc}

\title{%
  COMSOC: Homework 2 \\
  \large Silvan Hungerbuehler, 11394013}

\usepackage{mathptmx} % "times new roman"
\usepackage{amssymb}
\usepackage{amsmath, amsthm}
\usepackage{amsfonts}
\usepackage{enumitem}
\usepackage{verbatim}
\usepackage{hyperref}
\usepackage{comment}
\usepackage[margin=1in]{geometry}
\usepackage{float}
\usepackage{bm}
\usepackage{graphicx}
%\graphicspath{ {images/} }

\usepackage[normalem]{ulem}
\date{}
\begin{document}
\maketitle

\section*{Question 1}
\subsection*{a}
The antiplurality rule is characterized by the maximal(!) \textit{discrete distance} and the \textit{unanimous loser} consensus criterion. 
A candidate $x$ is an \textit{unanimous loser} with respect to some profile iff all voters rank $x$ last in their individual ballot.
\subsection*{b}
\textit{Cindarella-Swap Distance}:\\\
This measure of distance between two profiles - $\boldsymbol{R}$ and $\boldsymbol{R'}$ -  only takes into account swaps from one end of an individual ranking to the other. All swaps needed of candidates that are not extremely opposed are not counted by the measure. The \textit{Cindarella-Swap Distance} between $\boldsymbol{R}$ into $\boldsymbol{R'}$ is the minimal number of such extremely-opposed swaps needed to turn one into the other.
The antiplurality rule is characterized by the \textit{unanimous winner} consensus criterion and the maximal(!) \textit{Cindarella-Swap Distance}.
\subsection*{c}
If $n$ voters need to rank all candidates $X$ and each voter independently ranks the true winner - $a$ - last in her individual ballot with strictly smaller probability than any other of the candidates: $p_{last}(a) < p_{last}(b)$, for all $b \in X-\{a\}$. Then the probability that the \textit{anti-plurality} rule returns the correct decision increases monotonically in $n$ and approaches $1$ as $n$ goes to infinity.\\
For $a$ to win under the \textit{anti-plurality} rule it needs be placed last in the smallest fraction of all ballots: $\tfrac{\# \text{voters ranking a last}}{n}$. But by the law of large numbers $\tfrac{\#\text{voters putting $a$ last}}{n}$ converges to the probability that $a$ is ranked last by any individual voter: $p_{last}(a)$. By the construction of the noise model, $p_{last}(a)$ is strictly smaller than $p_{last}(b)$, for all $b \in X-\{a\}$, so as $n$ increases the probability of electing the correct candidate approaches $1$.

\section*{Question 2}
It is possible to find an SCF that satisfies the three desiderata for any $n=z*3$, where $z\in \mathbb{N}$. For all $n$ that are multiples of three no such SCF can be found.\\

\noindent One voting rule that has all the desired properties is the \textit{Plurality} Rule.\\
For $n\neq z*3$, $z\in \mathbb{N}$, this rule respects all desiderata: \textit{Anonymity}, for all voters have equal weight; \textit{neutrality}, for the candidates are treated equally; lastly, it can never end in a \textit{three-way tie}. To see this, consider what would be necessary to obtain a \textit{three-way tie} for any profile $\mathbf{R}$: all three elements of the candidate set $X$ would need to be in the top-position for the same number of ballots. So the total number of voters - and thus available top-positions - must be divisible by three for this to be possible. Yet this is precluded by the fact that we are only considering profiles where $n$ is not a multiple of three. 

\noindent It is impossible to find any SCF $\mathbf{F}$ that satisfies the three desiderata for three candidates and $n=z*3$ voters, where $z\in \mathbb{N}$.\\
Suppose some SCF satisfies \textit{anonymity} and \textit{neutrality}. For any profile with a number of voters that is a multiple of three we can construct a profile as the following:

\begin{align*}
a \succ b \succ c \\
c \succ a \succ b \\
b \succ c \succ a \\
\end{align*}
Suppose (w.l.o.g) that our SCF declares $a$ the winner. \textit{neutrality} allows us to swap $a$ with $b$ and then $b$ with $c$:
\begin{align*}
c \succ a \succ b \\
b \succ c \succ a \\
a \succ b \succ c \\
\end{align*}
Now, $c$ should be the winner (as it occupies the same positions in the profile as $a$ did previously).\\
Then, apply \textit{anonymity} to switch the order of the ballots in the profile. By assumption, this should not change the winner $c$. 
\begin{align*}
a \succ b \succ c \\
c \succ a \succ b \\
b \succ c \succ a \\
\end{align*}
But this is the original profile where $a$ had won. We have encountered a contradiction.\\
 
We have chosen $a$ arbitrarily as the initial winner, but the same argument holds for $b$ and $c$. Also, we reach the same contradiction if we assume that the initial profile yields a tie between any of the two candidates. SCF are defined so as to have at least one winner. So, the only non-contradictory situation is a three-way tie between all three candidates. Notice also that we can find a similar profile for which we reach the same contradictions for any $n$ that is a multiple of three; we would simply add a 'copy' of each voter to reach the desired number of voters.

\section*{Question 3}
How many voting rules?\\
For $x$ candidates and $m$ voters: The candidates could 'show up' at the election in $x!$ different orders (there are $x!$ ways they could receive their 'names'). Then, each individual voter has $x!$ many ways to order the named candidates in her individual ballot. Each of the $m$ voters independently chooses her own individual ordering. So, there are $x!*x!^m$ possible 'unordered profiles'. However, because the real profiles are ordered, there are are $m!$ more options for the ordered profiles. Thus, there are $x!*x!^m*m!$ possible ordered profiles. This is the cardinality of the domain of our social choice function (SCF). The cardinality of the codomain is simply $2^x-1$ - the cardinality of the power set minus the empty set. \\
There are $Z^Y$ many functions from a set with cardinality $Y$ to a set with cardinality $Z$. Thus there are $[2^x-1]^{x!x!^m*m!}$ SCF in general. For $m=n=3$ this makes $[2^3-1]^{3!*3!^3*3!}=7^{7776}$ many SCF. \\
How many of them anonymous?\\
There are now a mere $x!*x!^m$ elements in the domain to be considered (because a different ordering of individual rankings does not imply a different profile anymore). Therefore, there are $[2^x-1]^{x!*x!^m}$ SCF in general; particularly, $[2^3-1]^{3!*3!^3}=7^{1296}$ in our case.\\
How many neutral?\footnote{I know all about neutrality. See Appendix for proof.}\\
The order in which the candidates register for election does not matter here. This leaves us with $x!^m*m!$ many possible profiles. Specifically, then, there are $[2^3-1]^{3!*3!^3}=7^{1296}$ SCF.\\
How many anonymous and neutral?\\
Combining the two previous considerations leaves us with $x!^m$ possible profiles in general; in the present case this makes $7^{3!^3}=7^218$ SCF.\\
How many are resolute?\\
Since only exactly one candidate can be elected, the codomain's cardinality would be equal to the number of candidates: $x$. So there are $x^{(x!x!^m*m!)}$ possible SCF in general, or $3^{(3!*3!^3*3!)}=3^{7776}$ in our case.
How many are resolute and anonymous?\\
Combining the domain's cardinality under anonymity with the resolute codomain already determined yields generally $x^{(x!*x!^m!)}$ possible SCF - this makes $3^{(3!*3!^3)}=3^{1296}$ for $m=n=3$.\\
How many are resolute and neutral?\\
Similar considerations yield $x^{(x!^m*m!)}$ possible SCF in general, or  $3^{(3!^3*3!)}=3^{1296}$ for $m=n=3$.\\
How many are resolute, neutral and anonymous?\\
Combining all three constraints gives $x^{x!^m}$ many SCF in general. In particular, there are $3^{218}$ for $n=m=3$.
\section*{4}
\subsection*{a}
\textbf{Claim 1:} A SCF is \textit{paretian} and \textit{resolute} $\implies$ SCF is \textit{surjective}.\\
\textbf{Proof:} Take any $x\in X$ and construct $\mathbf{R}^*$ where $x$ is first in each individual ballot. For all other candidates in $X$, $y\in X-\{x\}$, $\mathbf{N}^{\mathbf{F}}_{x\succ y}=N$. By virtue of being \textit{paretian} the SCF thus cannot elect any of the $y\in X-\{x\}$. Because the SCF must return something by definition $x$ will be elected. We chose the $x$ arbitrarily, thus the SCF is \textit{surjective}.\\

\noindent \textbf{Claim 2:} There exists a SCF that is \textit{surjective} and \textit{resolute} but no \textit{paretian}.\\

\noindent \textbf{Example:} The \textit{antiplurality} rule (lexicographic tie-breaking) is \textit{surjective} and \textit{resolute} but fail to be \textit{paretian}.\\
It is \textit{surjective} because for any $x\in X$ we can find a profile $\mathbf{R}$ where all voters rank $x$ last in their ballot. Hence, all $x\in X$ can be elected. Because we can find such an $\mathbf{R}$ this rule is not \textit{paretian}. If every voter ranks $x$ last then any of the other $y\in X-\{x\}$ would be preferred by all voters. Nonetheless, $x$ is elected which violated the \textit{paritian} condition. The rule is certainly resolute because of the tie-breaking rule.

\subsection*{b}
The \textit{Anti-dictatorship} is such a rule. It picks some voter, say the first in the profile, and elects whatever her least preferred option happens to be.\\
surjective, independent, non-dictatorial, resolute
This is \textit{surjective}, for any candidate could be placed last in the first ballot of some profile. Therefore any candidate can be elected.\\
It is resolute because there always only one last position in the anti-dictator's ballot. The candidate in that position is the sole winner.\\
It is non-dicatorial because no voter has the garantuee that her first option will be elected.\\
It is independent. If some candidate $x$ wins in a profile $\mathbf{R}$, then this is because it occupies the last position in the anti-dicator's individual ballot. So, no other candidate $y$ can win if we change $\mathbf{R}$ to $\mathbf{R}'$ but keep the relative ranking of $x$ and $y$ as it is in every individual ballot. Because to change the winner to $y$, it would have to be dispreferred in the anti-dictator's ballot.

%http://www.sciencedirect.com/science/article/pii/0022053172900518?via%3Dihub

\section*{Appendix}

\end{document}