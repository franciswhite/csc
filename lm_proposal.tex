\documentclass{article}
\title{Give us 1.5 Monnie\$\$\$\\
\large Research Proposal}
\date{}
\author{Julia Turska, Silvan Hungerb{\"u}hler}
\usepackage{enumitem}
\usepackage{verbatim}
\usepackage{hyperref}
\usepackage{comment}
\usepackage[utf8]{inputenc}
\usepackage[english]{babel}
\usepackage{amsmath}

\usepackage{biblatex}
\addbibresource{lm_references.bib}
\usepackage{csquotes}
\begin{document}
\maketitle
\section{Aim of the Research - Julia}
A large body of findings of studies on reasoning suggest that people often perform inference in a non-normative way, for instance support ungrounded conclusions or do not know how to properly check validities of rules (references). One of the available explanations of this is proposed by Keith Stanovich, namely: (on the relationship between reasoning and rationality) there is a normative/descriptive gap between the rules of logic and actual human reasoning and hence there is much room for \textbf{improvement by suitable education}.\\ 
\indent This turns out to be problematic, as there is no proper model of what type of knowledge acquisition should be considered here as suitable education? Is it the familiarity with the rules or being accustomed to solving this type of problems? Is there a way to make this determination? \\
\indent The idea which comes to mind to study how experts in this domain, namely logicians,  approach reasoning tasks. The prediction of Stanovich's theory would be that they perform much better. Why should it be the case? Is it the fact that they declaratively know the proper rules of reasoning which makes them better at solving reasoning puzzles or is it the skills (procedural knowledge) which they have obtained throughout their training? Our proposal is to thoroughly study the structure of expert knowledge in logic, and then try to implement this knowledge structure in an ACT-R model. This would be followed by a study in the novice/expert paradigm, thanks to which we could determine the empirical plausibility of the ACT-R model.  \\ %these are 266 words
\textbf{Less than 200 words}

\section{Prior Research - Silvan(Lowe, Elicitation)/Julia(Knowledge Model)}
\textbf{comprehensible literature review of the research field, 
converging upon the research questions;  not more than 500 words\\
•Describe the state of affairs in the current research field based on the existing body of literature \\
• Elaborate on the theoretical framework\\
• Clarify how previous research leads to the research questions of the current proposal }\\
\subsection{Theories of reasoning and types of knowledge}
Sub-optimal performance. a broad spectrum of explanations
Stanovich: individual differences, factors like age, cognitive capacities and so on. 
Improvement through training--> 
\subsection{Expert study paradigm}
Expert studies are an established paradigm in science \cite{shaw1990modeling}. A variety of cognitive and non-cognitive tasks have been studied with respect to the differences between novices' and experts' \cite{carter1987processing}\cite{carter1988expert}\cite{peskin1998constructing} \cite{means1985star}. Such studies have at times aided the construction of explicit models to explain expert performance in the respective field. This, in turn, allows precise predictions which can then be subjected to empirical testing.

There is currently no model that explicitly captures the content of expert knowledge in the domain of reasoning. Although there exist models of knowledge in the related field of mathematics \cite{lowe2010skills}, as far as we are aware no such effort has been made for logical reasoning to date. Our modeling of knowledge concerning reasoning is thus novel, albeit related to . 

There exists a series of established practices and methods concerning the \emph{elicitation} of expert knowledge for general domains. To our knowledge, there is no canonical method to systematically inquire into experts' mathematical or reasoning skills. Our acquisition of the knowledge that these experts use to solve their respective problems therefore relies on techniques in other domains, for example, in engineering or industrial management \cite{ford1997expert} \cite{cooke1994varieties}. A good deal of the literature on elicitation methods focuses on the acquisition of knowledge that is to be implemented in so-called \emph{expert systems} - computer applications solving complex problems \cite{shadbolt1995knowledge}. This part of the literature is particularly suitable to us, as are ultimate intention in building a conceptual model of expertise in logical reasoning consist in turning it into one at David Marr's algorithmic level. % \cite{marr}
\subsection{A Computational Model of Logical Knowledge Acquisition}
\section{Key Questions - Silvan}
\textbf{this is the essence of the proposal. Hypotheses should be 
derived from the key question; the relation between theory and research
hypotheses should be clearly specified; not more than 500 words\\
a) Formulate the research questions based on previous research\\
b) Translate the specific research questions into research hypotheses }\\

Is performance on logical reasoning task to be explained by \emph{procedural} knowledge - presumably consisting of an interaction of general intelligence and prior experience with similar problems - or by \emph{factual} knowledge - knowing rules of formal logic?
Related to this issue is the question to what extent the logical reasoning of an individual can improve through suitable training. If it were possible to determine the extent in which performance is aided by declarative knowledge, then the teaching of logical reasoning would certainly benefit. It is thus of interest to determine the influence of expertise in formal symbolic logic on human performance in logical reasoning tasks.



As there is currently no literature on methods and techniques which professional logicians use to tackle logical problems, it is pertinent to inquire into the strategies which experts in the field of formal logic employ to tackle reasoning problems.
\section{Procedure}
\textbf{not more than 500 words \\
a) Operationalize the research questions in a clear manner into a modeling strategy\\
b) Describe the procedures for conducting the research (and collecting
 the data) }
 Experiment:\\
 1. expert/novice: (feed conceptual part)\\
a. performance ona particular reasoning task (selection task (MP)? $\implies$ evaluation of Stanovic's hypothesis\\
 b. categorization task (reference?) $\implies$ declarative/procedural knowledge, deep /surface structure\\
 2. experimental phase: (feeds ACT-R part) - THIS IS ONLY A SKETCH! PLZ NOT THE BELT AGAIN!\\
 Control Group: left untrained\\
 Training Group 1: logical training: declarative teaching paradigm\\
 Training Group 2: logical training: procedural paradigm\\
 Training Group 3: logical training: have both trainings\\
 Group Selection: screened for the factors mentioned by Stanovic
 \section{Intended Results - Julia}
\textbf{clarify what the implication of possible outcomes would be 
(per hypothesis) for the research questions as well as for the theory; not more than 500 words \\
• To what extent can the proposed research provide an answer to the re
search 
questions? \\
• What are the (alternative) interpretations if the results do (not) 
match the expectations?} 
There are principally three findings we aim for:\\
First, training in formal logic should have a positive effect on performance on reasoning tasks. In case such an effect is indeed observable, our experimental set-up allows us to isolate the precise content of this t. That is, we are able to distinguish between improvements rooted in acquiring declarative knowledge in the form of logical rules and improvements attributable to procedural knowledge.

Second, educational strategies.

Third, expert knowledge paradigm extension.

\printbibliography

\end{document}
