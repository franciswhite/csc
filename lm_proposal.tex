\documentclass{article}
\title{Give us Monnie\$\$\$\\
\large Research Proposal}
\date{}
\author{Julia Turska, Silvan Hungerb{\"u}hler}
\usepackage{enumitem}
\usepackage{verbatim}
\usepackage{hyperref}
\usepackage{comment}
\usepackage[utf8]{inputenc}
\usepackage[english]{babel}
\usepackage{amsmath}

\usepackage{biblatex}
\addbibresource{lm_references.bib}
\usepackage{csquotes}
\begin{document}
\maketitle
\section{Aim of the Research - Julia}
The findings of empirical studies on reasoning suggest that people often do not follow classical rules of inference, for instance support ungrounded conclusions or do not know how to properly check validities od rules (references). One of the available explanations of this is proposed by Stanovich, namely: (on the relationship between reasoning and rationality) there is a normative/descriptive gap between the rules of logic and actual human reasoning, as the latter falls short of prescriptive standards, which are themselves subnormal; there is therefore much room for \textbf{improvement by suitable education}.\\ 
\indent This turns out to be problematic, as there is no proper model of what type of knowledge acquisition should be considered here as suitable education? Is it the familiarity with the rules or being accustomed to solving this type of problems? Is there a way to make this determination? \\
\indent The idea which comes to mind to study how experts in this domain, namely logicians,  approach reasoning tasks. The prediction of Stanovich's theory would be that they perform much better. Why should it be the case? Is it the fact that they declaratively know the proper rules of reasoning which makes them better at solving reasoning puzzles or is it the skills (procedural knowledge) which they have obtained throughout their training? Our proposal is to thoroughly study the structure of expert knowledge in logic, and then try to implement this knowledge structure in an ACT-r model. This would be followed by a study in the novice/expert paradigm, thanks to which we could determine the empirical plausibility of the ACT-r model.  \\
\textbf{Less than 200 words}

\section{Prior Research - Silvan(Lowe, Elicitation)/Julia(Knowledge Model)}
\textbf{comprehensible literature review of the research field, 
converging upon the research questions;  not more than 500 words\\
•Describe the state of affairs in the current research field based on the existing body of literature \\
• Elaborate on the theoretical framework\\
• Clarify how previous research leads to the research questions of the current proposal }\\

Expert studies are an established paradigm in science \cite{shaw1990modeling}. A variety of cognitive and non-cognitive tasks have been studied with respect to the differences between novices' and experts' \cite{carter1987processing}\cite{carter1988expert}\cite{peskin1998constructing} \cite{means1985star}. Such studies have at times aided the construction of explicit models to explain expert performance in the respective field. This, in turn, allows precise predictions which can then be subjected to empirical testing.

There is currently no model that explicitly captures the content of expert knowledge in the domain of reasoning. Although there exist models of knowledge in the arguably related field of mathematics \cite{lowe2010skills}, no such effort has been made for logical reasoning to date.

Nonetheless, there exists a series of established practices and methods of elicitation

\section{Key Questions - Silvan}
\textbf{this is the essence of the proposal. Hypotheses should be 
derived from the key question; the relation between theory and research
hypotheses should be clearly specified; not more than 500 words\\
a) Formulate the research questions based on previous research\\
b) Translate the specific research questions into research hypotheses }\\

It is often assumed that performance on reasoning task is explained by procedural knowledge, presumably consisting of an interaction of general intelligence and prior experience with similar problems. Factual knowledge, however, plays no role in such explanations. It is thus of interest to determine the influence of expertise in formal symbolic logic on the performance of logical reasoning tasks.

As there is currently no literature on methods and techniques which professional logicians use to tackle logical problems, it is pertinent to inquire into the strategies which experts in the field of formal logic employ to tackle reasoning problems.
\section{Procedure}
\textbf{not more than 500 words \\
a) Operationalize the research questions in a clear manner into a 
modeling 
strategy\\
b) Describe the procedures for conducting the research (and collecting
 the data) }
 \section{Intended Results - Julia}
\textbf{clarify what the implication of possible outcomes would be 
(per hypothesis) for the research questions as well as for the theory; not more than 500 words \\
• To what extent can the proposed research provide an answer to the re
search 
questions? \\
• What are the (alternative) interpretations if the results do (not) 
match the expectations?} 
Logical training $\implies$ Better performance on logical task\\
educational strategies\\
expert knowledge paradigm extension

\section{References}
%\printbibliography

\end{document}
