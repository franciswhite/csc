\documentclass[10pt,a4paper]{article}
\usepackage[utf8]{inputenc}

\title{
   Summary of a talk by Herrade Igersheim for the Dutch Social Choice Colloquium\\ on April 21, 2017
\\
  \large Silvan Hungerbuehler, 11394013}

\usepackage{mathptmx} % "times new roman"
\usepackage{amssymb}
\usepackage{amsmath, amsthm}
\usepackage{amsfonts}
\usepackage{enumitem}
\usepackage{verbatim}
\usepackage{hyperref}
\usepackage{comment}
%\usepackage[margin=1in]{geometry}
\usepackage{float}
\usepackage{bm}

\usepackage[normalem]{ulem}
\date{}
\begin{document}
\maketitle
\subsection*{The Death of Welfare Economics: History of a Controversy}
The following is a summary of Herrade Igersheim’s talk at the Dutch Social Choice Colloquium 2017. She is an economist by training and currently associated with the Bureau d'Économie Théorique et Appliquée (BETA) at the University of Strasbourg. Her talked revolved around a controversy between Kenneth Arrow, featured centrally in class, and the economist Paul Samuelson. Igersheim offered an historical account of the impact Arrow’s famous impossibility theorem had on the field of welfare economics in which Samuelson had contributed immensely. Specifically, she focused on the way in which Arrow’s theorem was received in the scientific community and on the effect it had on the standing of Samuelson’s work published priorly. The field of welfare economics is concerned with specifying a social welfare function (SWF) that ranks pairs of states a society could be in by assigning a real-valued utility to each of them. It can be regarded as the attempt to ethically order various states the world could be in, in order to determine which is the best rfeasible such state.

Samuelson’s work on welfare economics is a clarification of Abram Bergson’s, another economist, original ideas from 1938. The Bergson-Samuelson SWF (BS-SWF) put forward in this way would lose scientific standing after Arrow purportedly showed that a BS-SWF cannot exist under reasonable circumstances. Arrow showed that if we consider all logically possible preferences the individual of a society could have, then it is impossible to find a SWF which simultaneously satisfies the Weak Pareto and the Indepence principles while not being a dictatorship.

Igersheim’s talk, then, was concerned with the history and sociology of this particular point of research. For a technical discussion about the merits and defects of Arrow’s and Samuelson’s respective conception of social welfare she referred to other researchers' specifically, Fleurbaey and Mongin. \footnote{Cf. \href{http://econpapers.repec.org/article/sprsochwe/v_3a25_3ay_3a2005_3ai_3a2_3ap_3a381-418.html}{Fleurbaey \& Mongin, 2005}}
The story Igersheim sketched goes thus: While Arrow initially advocated that his theorem precludes the possibility of constructing a BS-SWF, Samuelson ardently denied that Arrow’s theorem was applicable to his work. Perhaps the central reason given to deny its validity for welfare economics is the following: While Arrow’s SWF meets the condition of a universal domain (all logically possible profiles are considered), the BS-SWF is only used to find the optimal state for a given profile. Furthermore, Samuelson was suspicious of the neturality axiom

Interestingly, the two could not sort out their differences of opinion all their lives, despite repeated attempts and the fact that they were close friends. Igersheim believes that Arrow himself grew less convinced over time about the relationship of his work to the BS-SWF. However, initially he was at most ambiguous about the issue and thereby helped to sink the ship Samuelson and Bergson had built. According to Igersheim, most scholars seem to have accepted that Arrow’s theorem rightfully buried the welfare economist’s idea of SWF and thus paved the way for the scientific field of Social Choice. Starting at the end of the 1970ies the Social Choice community was institutionalized; first proper Journals were published -  \href{http://link.springer.com/journal/355}{Social Choice and Welfare} - and groups consolidated for the cause - \href{http://www.unicaen.fr/recherche/mrsh/scw}{The Society for Social Choice and Welfare}.
\\
\\

\noindent \textbf{Word Count: 552}
\end{document}